% OCR draft from PDF pages 220-234 (book chapter 7).
\chapter{A Simulation Environment}
\label{chap:simulation-environment}

\section{Introduction}
\label{sec:simenv-intro}

This chapter presents a user-level view of the SMPL simulation environment.
SMPL includes:
\begin{itemize}
\item the \texttt{smpl} subsystem from Chapter 2,
\item debugging/analysis/reporting tools,
\item an interactive run-time interface called \texttt{mtr}.
\end{itemize}

In the implementation described (SMPL/PC), \texttt{mtr} acts as the interactive
front-end between user, simulation model, and support modules.

\begin{figure}[ht]
\centering
\fbox{\parbox[c][5cm][c]{0.86\textwidth}{\centering Figure 7.1 placeholder (SMPL simulation environment)}}
\caption{The SMPL Simulation Environment}
\label{fig:smpl-environment}
\end{figure}

Main optional modules:
\begin{itemize}
\item \textbf{dump}: event list, queue contents, facility users/status.
\item \textbf{table}: table definition, entry, reports, and plots.
\item \textbf{bma}: batch means analysis for run-length control.
\item \textbf{parms}: named parameter definition and access.
\item \textbf{dis}: graphics display (time series and parameter plots).
\end{itemize}

\section{The Run-Time Interface}
\label{sec:simenv-mtr}

The simulation program chooses whether \texttt{mtr} is active (via
\texttt{smpl()} options), so production runs can bypass interaction.

When enabled, \texttt{mtr} pauses:
\begin{enumerate}
\item after initial \texttt{smpl} initialization (for parameter/table setup),
\item after first event scheduling (when facilities exist; useful for display
  setup).
\end{enumerate}

During execution, \texttt{mtr} can pause on user input, display completion,
model breakpoints, or errors. While running, it monitors each event boundary:
breakpoint checks, active analysis/display updates, and function-key polling.

\begin{figure}[ht]
\centering
\fbox{\parbox[c][5cm][c]{0.86\textwidth}{\centering Figure 7.2 placeholder (SMPL/PC function keys)}}
\caption{SMPL/PC Function Keys}
\label{fig:smpl-function-keys}
\end{figure}

Representative controls include pause/continue, trace toggles, dump/report
display, table/bma/dis setup, parameter display/edit, breakpoint control, stream
selection, and optional user-defined display function invocation.

\section{Parameters}
\label{sec:simenv-parameters}

SMPL parameters are named real-valued variables registered by the model and
shared with \texttt{mtr} and modules via the \texttt{parms} registry.

Benefits:
\begin{itemize}
\item reduces custom input/report coding in models,
\item allows keyboard display/modification of selected model variables,
\item provides direct inputs for table/bma/dis,
\item supports parameter-based breakpoints.
\end{itemize}

\section{The \texttt{dump} Module}
\label{sec:simenv-dump}

\texttt{dump} displays the current simulator state:
\begin{itemize}
\item per-facility server state and reserving token/priority,
\item queue contents (event/token/priority/time-left),
\item queue activity timestamps,
\item event list entries and last-caused event.
\end{itemize}

It can be invoked by model call, function key, or automatically on simulator
error.

\begin{figure}[ht]
\centering
\fbox{\parbox[c][5cm][c]{0.86\textwidth}{\centering Figure 7.3 placeholder (dump output example)}}
\caption{\texttt{dump} Module Output Example}
\label{fig:smpl-dump-example}
\end{figure}

\section{The \texttt{table} Module}
\label{sec:simenv-table}

\texttt{table} provides histogram-like collection/reporting:
\begin{itemize}
\item define value range and interval count,
\item enter observations,
\item report mean/stddev and interval frequencies/cumulative frequencies,
\item optionally plot the table.
\end{itemize}

Tables can be model-defined (entries by function call) or \texttt{mtr}-defined
(entries by parameter/count observation). A single \texttt{mtr}-driven table is
supported to keep monitoring overhead low.

\section{The Batch Means Analysis Module}
\label{sec:simenv-bma}

\texttt{bma} estimates run length for a target relative confidence-interval
half-width on a discrete-time output variable.

Workflow:
\begin{enumerate}
\item discard initial observations (\emph{warm-up}),
\item collect initial batches and test lag autocorrelation,
\item if dependent, merge adjacent batches (doubling batch size),
\item when independence is acceptable, keep adding batches until target
  relative half-width is met,
\item optionally reduce final batch count (to improve coverage) while
  preserving target accuracy.
\end{enumerate}

Input setup (from \texttt{mtr} or model code) includes output/count parameters,
warm-up count, initial batch size/count, target relative half-width, confidence
level, lag-count, significance level, and max-observation limit.

\begin{figure}[ht]
\centering
\fbox{\parbox[c][5cm][c]{0.86\textwidth}{\centering Figure 7.4 placeholder (bma input/output displays)}}
\caption{\texttt{bma} Input and Output Displays}
\label{fig:smpl-bma}
\end{figure}

The chapter notes that batch-means methods are empirical tools and should be
used with care; it references broader output-analysis comparisons.

\section{The Graphics Display Module}
\label{sec:simenv-dis}

\texttt{dis} provides interactive plots, including:
\begin{itemize}
\item facility utilization vs.\ time,
\item average queue length vs.\ time,
\item current queue length vs.\ time,
\item parameter vs.\ time,
\item parameter vs.\ parameter/count.
\end{itemize}

Plots are configured from an input panel (option-specific parameters such as
facility/parameter IDs, axis ranges, time span, plotting mode). Data can also be
written to file for external plotting.

\begin{figure}[ht]
\centering
\fbox{\parbox[c][4.5cm][c]{0.86\textwidth}{\centering Figure 7.5 placeholder (\texttt{dis} step-mode queue plot)}}
\caption{\texttt{dis} ``step mode'' Plot}
\label{fig:smpl-dis-step}
\end{figure}

Typical uses: warm-up estimation, ad hoc run-length checks, dynamic behavior
inspection, and bug discovery via live trend observation.

\section{Summary}
\label{sec:simenv-summary}

The chapter's core point is productivity: parameterization plus interactive
control reduce code/compile/debug cycles and support rapid model experimentation.
With \texttt{mtr}, \texttt{parms}, and optional analysis/display modules, many
studies can be run with little or no model recoding. The implementation details
are system-dependent mostly in UI/display glue, making portability practical.
